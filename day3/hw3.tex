\documentclass[12pt]{article}
\usepackage{graphicx}
\usepackage{psfrag}
\usepackage{epsfig}
\usepackage{subfigure}
\usepackage{amssymb,amsmath}

%\usepackage{doublespace}

\textheight 23cm \topmargin -1cm \leftmargin 0cm

\marginparwidth 0mm    % Largeur des notes marge de droite
\textwidth 16.5cm      % Largeur du texte
\hsize \textwidth      % Longueur d'une ligne
\advance \hsize by -\marginparwidth
\oddsidemargin -4mm    % Marge gauche pages de droite - 1 inch (2.54 cm)
\evensidemargin \oddsidemargin % Idem pour les pages de gauche

\advance\hoffset by 5mm % Pour corriger un decalage residuel sur la gauche

\renewcommand{\hat}{\widehat}
\newcommand{\R}{\mathrm{I\!R\!}}
\newcommand{\C}{\mathrm{I\!\!\!C\!}}
\newcommand{\sinc}{\mbox{sinc}}
\newcommand{\diag}{\mbox{diag}}
\newcommand{\Tr}{\mbox{Tr}}

% commands
\newtheorem{theo}{Theorem}
\newtheorem{proposition}{Proposition}
\newcommand{\nc}{\newcommand}
\nc{\RR}{\mbox{\rm I$\!$R}} \nc{\dsp}{\displaystyle}
\nc{\Div}{\mbox{\rm div }} \nc{\beequ}{\begin{equation}}
\nc{\barr}{\begin{array}} \nc{\earr}{\end{array}}
\nc{\eequ}{\end{equation}} \nc{\BFF}{\hbox{\boldmath{${\cal
F}$}}^{(j)}({\bf y}^s,t)}

\nc{\hr}{\widehat{r}} \nc{\hh}{\widehat{h}} \nc{\hs}{\widehat{s}}
\nc{\hn}{\widehat{n}} \nc{\hw}{\widehat{w}} \nc{\hd}{\widehat{d}}
\nc{\hG}{\widehat{\Gamma}} \nc{\om}{\omega} \nc{\yy}{{\bf y}}
\nc{\ys}{{\bf y}^s} \nc{\yo}{{\bf y}_0} \nc{\xp}{{\bf x}_p}
\nc{\xr}{{\bf x}_r} \nc{\co}{{\cal O}}
\renewcommand\labelenumi{\alph{enumi})}
\begin{document}
\hfill August 10, 2016

\begin{center}
{\Large \textbf{MIMO Wireless}}
\\

\textbf{Practice Set 3 }
%\textbf{Due: Thursday January 21, 2010, in-class}
\end{center}
\paragraph{Problem 1} 
Lets the transmit symbol vector $\tilde{{\bf X}} = [1 + j,-1 + j]$ (which is not normalized). Find received symbol vector ${\bf Y}$ (ignore the the noise) and ${\bf G}_{ZF}$, where  
\begin{eqnarray*}
{\bf H} &=& \left[
\begin{array}{cccc}
-0.8256 + 0.8831i & 0.8025 + 0.4420i \\
-0.2678 + 0.3977i & 1.2196 - 0.0895i
\end{array}
\right]
\end{eqnarray*}
Verify your results.


\paragraph{Problem 2} {\it Diversity Gain}\\
We are given a $(1 \times M)$ SIMO channel with each diversity branch modeled as
\begin{equation*}
y_i = \sqrt{E_s} h_i s + n_i,\,\,i=1,\cdots,M
\end{equation*}
where $n_i$ is additive ZMCSCG noise with variance $ N_0$.
\begin{enumerate}
\item With Maximum Ratio Combining (MRC), find the received SNR ($\eta$). 
\item Assuming Maximum Likelihood (ML) detection at the receiver, show that we can calculate the probability of symbol error $P_e$ as
\begin{equation*}
P_e \leq \overline N_e Q \left(\sqrt{  \frac{  \rho  d_{min}^2 \lVert h \rVert^2}{2}}\right)
\end{equation*}
with $\overline N_e$ being the average number of nearest neighbors and $d_{min}$ being the minimum distance between points in the underlying scalar constellation. 
%\item Derive an upper bound for the average probability of error with Rayleigh fading.  Note that $|h_i|^2$ is an exponential random variable with mean 1.
%\item What happens in the high SNR regime?
\end{enumerate}

\paragraph{Problem 3}{\it Dominant eigenmode transmission.} \\
For an $M_T\times M_R$ MIMO system with CSIT, consider a transmission scheme that uses all available transmit power in the strongest mode.
\begin{enumerate} 
\item Show that the expected array gain is given by
\begin{equation*}
AG = {\cal E} \{  \lambda_{max} \}.
\end{equation*}
%and find a lower-bound and upper-bound for $\lambda_{max}$.
\item Using the result from (a) and the Chernoff bound, derive lower and upper bounds for the average probability of error $\overline P_e$ in the high SNR regime. Assume an independently  Rayleigh-fading channel matrix.  The following may be useful:
\begin{eqnarray*}
\Vert{\bf H}\Vert_F^2 &=& \sum_{i=1}^r \lambda_i \\
\frac{\sum_{i=1}^r \lambda_i}{r} &\le&  \lambda_{max} \ \le\  \sum_{i=1}^r \lambda_i\end{eqnarray*}
Here, $r$ is the rank of ${\bf H H}^H$.
\item What can we conclude about the diversity performance of dominant mode transmission?
\end{enumerate}


\paragraph{Problem 4}{\it Comparison of Diversity Schemes (MIMO)}\\
We need to send reliable data to a receiver through a Rayleigh fading channel and want to compare three different diversity schemes. Sweep the SNR from 0 to 15 dB and assume independently Rayleigh fading channels. We are using BPSK modulation, i.e., $d_\text{min} = 2$ and $\overline N_e=1$.
\begin{enumerate}
\item Assume  a SISO link where the received signal $y $ is given as
$
y = \sqrt{E_s} hs + n
$
with noise power $\mathcal E\{\vert n \vert^2 \} = N_0$.
Plot the average probability of error $\overline P_e$ over $\rho = E_s/N_0$.  (Recall that $Q(x)=1/2 \cdot \mathrm{erfc}(x / \sqrt{2})$.)
\item On the same graph, plot the average probability of error $\overline P_e$ over $\rho$ assuming a $2 \times 2$  MIMO system with no channel knowledge and  Alamouti coding. 
\item Finally, plot $\overline P_e$ over $\rho$ assuming a $2 \times 2$ MIMO system with dominant eigenmode transmission. 


\end{enumerate}
\paragraph{Problem 5} 
Consider a $M \times M$ MIMO channel with ZMCSCG elements
of unit variance.  

\begin{enumerate}
\item For $M=2$, plot ergodic channel capacity and 10\% outage capacity over $\mathrm{SNR}$ with and without channel state information at the transmitter. How does channel state information affect the capacities at low and high SNR, respectively?

\item For $\mathrm{SNR}=10$\,dB, plot ergodic capacity over the number of antennas $M$, for $M \in \{2,4,6,8\}$. How does the value of channel state information change with $M$?
\end{enumerate}


\end{document}
